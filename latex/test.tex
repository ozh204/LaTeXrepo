\documentclass[a4paper,11pt]{article}
\usepackage{latexsym}
\usepackage{polski}
\usepackage[UTF8]{inputenc}
\usepackage{amsmath}
\usepackage{hyperref}
\hypersetup{
    colorlinks=true,
    citecolor=black,
    filecolor=black,
    linkcolor=blue,
    urlcolor=black,
}
\author{Piotr~Laskowski}
\title{Liczby~Zespolone}
\frenchspacing
\begin{document}
\maketitle
\tableofcontents
\section{Liczby zespolone}
 Liczby będące elementami rozszerzenia ciała liczb rzeczywistych o~jednostkę urojoną i, tj.~pierwiastek wielomianu $x^2+1$ (innymi słowy, jednostka urojona spełnia równanie $i^2 = -1$). Każda liczba zespolona z~może być zapisana w postaci $z=a + bi$, gdzie a, b są pewnymi liczbami rzeczywistymi, nazywanymi odpowiednio częścią rzeczywistą oraz częścią urojoną liczby~z.
\section{Postać algebraiczna (kanoniczna)}
Każdą liczbę zespoloną z można zapisać w postaci $$z=a+bi$$, gdzie a i b są pewnymi liczbami rzeczywistymi oraz i jest tzw. jednostką urojoną, tj. i jest jednym z dwóch elementów zbioru liczb zespolonych, spełniającym warunek $i^2=-1$ (drugim elementem jest -i). Spotyka się czasami zapis $i=\sqrt{-1}$, który nie jest formalnie poprawny ze względu na fakt, że również $(-i)^2=-1$, jest on jednak uznawany za pewien skrót myślowy i powszechnie akceptowany.

Postać $z=a+bi$ nazywana jest postacią algebraiczną (albo kanoniczną) liczby zespolonej z.

Dla liczby $z=a+bi$ definiuje się jej
\begin{itemize}
\item część rzeczywistą (łac. pars realis) jako re $z = a$ (inne oznaczenia: $\Re z, Re, z$),
\item część urojoną (łac. pars imaginaria) jako im $z = b$ (inne oznaczenia: $\Im z, Im, z$).
\end{itemize}
Przykładowo liczba $\textbf{7 - 5i}$ jest liczbą zespoloną, której część rzeczywista wynosi 7, a część urojona -5. Liczby rzeczywiste są utożsamiane z liczbami zespolonymi o części urojonej równej 0.

Liczby postaci $z = 0 + bi$ nazywa się liczbami urojonymi.
\subsection{Zapis alternatywny}
W zastosowaniach fizycznych, elektrycznych, elektrotechnicznych itp. zapis $z = a + bi$ może okazać się mylący z powodu wykorzystywania w tych dziedzinach litery \textrm{\Large{i}} do innych celów, np. chwilowego natężenia prądu elektrycznego. Dlatego też stosuje się zapis niepowodujący podobnych kłopotów, mianowicie $z = a + jb$, w którym to j oznacza jednostkę urojoną.
\subsection{Równość}
Dwie liczby zespolone są równe wtedy i tylko wtedy, gdy ich części rzeczywiste i urojone są sobie równe. Innymi słowy, liczby zespolone postaci a + bi\; oraz c + di\; są sobie równe wtedy i tylko wtedy, gdy a = c\; oraz b = d\;.

\subsection{Działania[}
Dodawanie, odejmowanie i mnożenie liczb zespolonych w postaci algebraicznej wykonuje się tak samo jak odpowiednie operacje na wyrażeniach algebraicznych, przy czym i^2 = -1:\;

(a + bi) \pm (c + di) = (a \pm c) + (b \pm d)i\;
(a + bi)(c + di) = ac + (bc + ad)i + bd i^2 = (ac - bd) + (bc + ad)i\;.
Aby podzielić przez siebie dwie liczby zespolone, wystarczy pomnożyć dzielną i dzielnik przez liczbę sprzężoną do dzielnika (analogicznie do usuwania niewymierności z mianownika w wyrażeniach algebraicznych):

{a + bi \over c + di} = {(a + bi)(c - di) \over {(c + di)(c - di)}} = {(ac + bd) + (bc - ad)i \over c^2 + d^2}
\end{document} 